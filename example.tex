\documentclass[twocolumn]{article}
%%%%%%%%%%%%%%%%%%%%%%%%%%%%%%
%%%%%%%%%% Preamble %%%%%%%%%%
%%%%%%%%%%%%%%%%%%%%%%%%%%%%%%

% us -> a4paper, also added 3cm margins
\usepackage[a4paper, margin=3cm]{geometry}

% makes abstract look good
\usepackage{abstract}

% date, bibliography... will instead be german
\usepackage[ngerman]{babel}

% extra math..
\usepackage{amsmath, amsfonts, amssymb}

% colors
\usepackage{xcolor}

% nice boxes
\usepackage[most]{tcolorbox}

% nice tables
\usepackage{booktabs}

% normal tables
\usepackage{tabularx}

% no indent after paragraph
\usepackage{parskip}

% import for \incfig
\usepackage{import}

% for authors
\usepackage{authblk}
\setcounter{Maxaffil}{0}
\renewcommand\Affilfont{\itshape\small}



% for bibliograpy
\usepackage{biblatex}
\usepackage{csquotes}

% \bib to add bib package
\newcommand\bib{\usepackage{biblatex}}

% \bibfile{file-without-ending} for bibfile
\newcommand{\bibfile}[1]{\bibliography{#1.bib}}


% to draw
\usepackage{tikz}
% to plot
\usepackage{pgfplots}
\pgfplotsset{compat=1.18} % mandatory 


% for inkscape figure manager
% > https://github.com/d-rens/inkscape-figures
\newcommand{\incfig}[2][1]{%
    \def\svgwidth{#1\columnwidth}
    \import{./figures/}{#2.pdf_tex}
}


% mostly got this of https://github.com/XYQuadrat
\tcbset {
    base/.style={
        arc=0mm, 
        bottomtitle=0.5mm,
        boxrule=0mm,
        colbacktitle=black!10!white, 
        coltitle=black, 
        fonttitle=\bfseries, 
        left=2.5mm,
        leftrule=1mm,
        right=3.5mm,
        title={#1},
        toptitle=0.75mm, 
    }
}

% Define new colorbox for research question
\newtcolorbox{ff}[1]{
    colframe=black,
    base={#1}
}


% to declare the task (delete after finishing)
\newtcolorbox{aufgabe}[1]{
    colframe=red,
    base={#1}
}





\definecolor{blau}{rgb}{0.4, 0.5, 1}
\newtcolorbox{mainbox}[1]{
  colframe=blau, 
  base={#1}
}

\newtcolorbox{subbox}[1]{
  colframe=black!40!white,
  base={#1}
}

\newtcolorbox{subsubbox}{
        arc=0mm, 
        bottomtitle=0.5mm,
        boxrule=0mm,
        colbacktitle=black!10!white, 
        colframe=black!30!white,
        coltitle=black, 
        fonttitle=\bfseries, 
        left=2.5mm,
        leftrule=1mm,
        right=3.5mm,
        toptitle=0.75mm, 
}



\bibliography{example.bib}


%% Note: If your authors are always the same you can move this to the preamble
% Authors, add more if you need to
\author[1]{Author}
\author[2]{Author}

% Author affiliations, use the number of the author to refer to them
\affil[1]{Author's institution}
\affil[2]{Another Author's institution}


\title{Physik Laborbericht Dokumentation}
\date{}

\begin{document}
\maketitle


\begin{abstract}
  In this document the author will try to explin the preamble he wrote.

  Further this abstract is to show that the preamble, or rather lab reports
  should contain an abstract, which is here made with the abstract package.
\end{abstract}

This is ideally becoming a package or class later, but for now a preamble will do.

The preamble is optimized to write physic lab reports, additions with it are
useful/needed packages. Another addition are the following tcolorboxes:
    



\section{tcolorboxes}
\label{sec:tcol}


\begin{ff}{Forschungsfrage}
    ff, for Forschungsfrage, black colframe to stand out, use this only once to define the reasearch question.
\label{Forschungsfrage}
\end{ff}


\begin{aufgabe}{Aufgabe}
    aufgabe, to define the questions given in the tasksheet, these are to be
    removed later, just to guide while writing, thus they were made red.
\label{aufgabe-1}
\end{aufgabe}


\begin{mainbox}{Mainbox}
    mainbox, this is for important things, maybe something like the following:

% Nested mainbox
\begin{mainbox}{Erkenntnis}
    Daraus ergibt sich
    \[ F_{el} \propto \frac{1}{r^2} .\] 
\end{mainbox}
\end{mainbox}


\begin{subbox}{subbox}
    subbux, for less important things, that are still highlight worthy.
\end{subbox}


\begin{subsubbox}
    subsubbox, to highlight something, but without title.
\end{subsubbox}

here are the boxes nested
\begin{mainbox}{Here you}
    \begin{ff}{can see}
        \begin{aufgabe}{a lot of}
            \begin{subbox}{nested}
                \begin{subsubbox}
                    boxes
                \end{subsubbox}
            \end{subbox}
        \end{aufgabe}
    \end{ff}
\end{mainbox}



\section{Citing}
One cites as usual, e.g. ``The \TeX{}book''~\cite{knuth1986texbook}\\
There is no addition yet, it already makes very much sense.



\vfill
% here the bibliography is printed 
\printbibliography

\end{document}
